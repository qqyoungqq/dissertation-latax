\chapter{Technical Background}
\section{Global Positioning System}
The Global Positioning System (GPS) has developed by the U.S Department of Defense (DoD) to provide civilian and military users with worldwide positioning, navigation and timing services.  Here, a brief introduction about principles of GPS and errors sources of GPS position measurement is presented.  For detailed studies, the reader is referred to special literatures, \citet[e.g.,][]{dixon1991chpt2,mao1999noise,hofmann2001gps}
  
\subsection{Struture}
The GPS system consists of three segments: the space segment, the control segment, and the user segment.  The space segment consists of the GPS satellites that transmit radio signals to users.  The nominal GPS constellation consists 24 satellites that are equally spaced in 6 orbital planes with 4 satellites in each plane.  Orbital planes are 60 degree separated and inclined at about 55 degree respect to the equatorial plane.  Each satellite flies in medium Earth orbit at an altitude of about 20200 km and circles the Earth twice a day.  This constellation ensures users can view at least 4 satellites from any point on the earth. The control segment on the ground consists of a system of facilities that receive signals from the satellites, perform analysis to compute satellite orbital data (ephemerides) and clock corrections, and send ephemerides back to each satellite.  The user segment consists of the GPS receivers that receive the signals from the GPS satellite and convert them into three-dimensional position and time. 

\subsection{GPS signal}
Each GPS satellite transmit microwave signal on two carrier frequencies: L1 (1575.42 MHz) and L2 (1227.60 MHz).  Two pseudorandom noise (PRN) codes and navigation message are modulated into the carrier frequencies.  The Coarse/Acquisition (C/A) code is modulated into the L1 carrier. The Precision (P) code is modulated into the L1 and L2 carriers.  The P-code is encrypted into Y-code in the Anti-Spoofing (AS) mode, which denies unauthorized users to use it.  Note that as a major focus of the GPS modernization program, three new civil (L1C, L2C, L5) signals and a new military (M) signal are added to the L1 and L2 carriers. 

\subsection{GPS basic observations}
To determine three-dimensional position of a user, the GPS receiver should compute the range to at least four satellites combing with satellites positions at time of transmitting signals.  However, due to synchronism problem between satellite clock and receiver clock and other factors, the GPS receiver can only provide pseudorange measurements rather than the true geometry range.  GPS receivers are capable to provide two types of pseudorange observations: code pseudorange and carrier phase pseudorange observations.  The code pseudorange is obtained by multiplying the speed of light by the travelling time, where the travelling time is determined by correlating the received code (C/A or P(Y)) from the satellite with the replicas generated by the receiver.  The carrier phase pseudorange is obtained by multiplying the wavelength by difference between carrier phase from the satellite and the carrier phase generated by the receiver.  Carrier phase pseudorange is about two orders of magnitude precise than the code pseudorange, but the carrier phase observation is ambiguous by an integer number of cycles \cite[]{remondi1985global}.  In order to achieve millimeter-precision, the ambiguity problem is needed to be fixed.  Carrier phase ambiguity resolution has been studied by \citet{lichten1987strategies}, \citet{blewitt1989chpt2,blewitt2008chpt2} and \citet{bertiger2010chpt2}. 
 
\subsection{GPS error sources}
There are many sources of error that will contaminate the position measurements. Major GPS error sources are briefly discussed in below.

(i) Satellite clock and orbit errors

GPS satellite clock time should be synchronous with GPS time (the time scale used by the GPS system).  Error in the satellite clock has a major impact on the computed code pseudorange.  Satellite orbits error is the discrepancies between the predicted position of each satellite and the true satellite position, causing error in the computed position.  In this dissertation, we use the precise final orbits and adjusted clock products provided by the Jet Propulsion Laboratory to mitigate satellite clock and orbit errors.  Another source of precise satellite orbits and adjusted clock products is the International GNSS Service (IGS).

(ii) Atmospheric effects  

The GPS signals encounter both inonspheric refraction and tropospheric refraction when propagating through the atmosphere, causing propagation delay.  The ionospheric delay is frequency dependent and can be corrected by using dual frequency (L1/L2).  The tropospheric effect can be reduced by using an elevation mask to avoid receiving signals from satellites lower than a certain elevation.  The tropospheric delay must be modeled.  The tropospheric model consists of mean tropospheric parameters or measurements data (temperature, air pressure, water vapor) and a mapping function \cite[]{niell1996chpt2,bohm2006chpt2,boehm2006chpt2}.  

(iii) Multipath effects

Multipath effects means receiver antenna gets direct signal through straight-line path and reflected signals through multiple paths. Multipath effects on code observation are much larger than on the carrier-phase observation.  Due to the randomness, the multipath effects cannot be modeled.  But it can be reduced with relatively long time observations.  In our processing routine, we estimate observations every 24 hours to minimize the multipath error \cite[]{sella2002chpt2}. 

(iv) Antenna phase center offset and variation 

The electrical antenna phase center (APC) is the point in space where radio signal is received.  However, the position of APC varies depending on the intensity and direction of incoming signal.  Thus, antenna phase center variation (PCV) is defined as the difference between the APC of each measurement and the mean position of the electrical antenna phase center (MPC).   The antenna phase center offset (PCO) is defined as the difference between MPC and antenna reference point (ARP) given by the manufacture.  An absolute phase center correction model estimated by \citet{schmid2007intro} can be use to calibrate both GPS satellite and receiver antennas. 

Note that here only the major error sources are discussed.  Other factors such as receiver clock error, monument movement and software accuracy also cause errors in GPS position results.  Last but not least, deformations due to ocean tidal, earth tidal and atmospheric loading need to be modeled and removed so that the geophysical process of interest (e.g., ice mass loss) can be isolated from other geophysical processes.  
   
\section{Interferometric Synthetic Aperture Radar}

\section{Gravity Recovery and Climate Experiment}

\section{References}
\bibliographystyle{apalike}  
\bibliography{chpt2_ref}