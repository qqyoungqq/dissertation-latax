\chapter{Conclusions and Future Work}
\section{Conclusions}
This dissertation presents three studies that use satellite geodesy to study environmental and global change.  The main conclusions from these studies are described in Chapters Three, Four and Five and are summarized here. 

Chapter Three shows that short-term annual variations in coastal uplift in Greenland as measured by GPS are useful to study spatial and temporal changes in mass loss of the ice sheet.  Anomalously large uplift is observed at most GPS sites in 2010, indicating significant ice mass loss in 2010.  Comparison between GPS data and climatic data suggests that the anomalous melting in 2010 is caused by a combination of warm air and warm sub-surface ocean water.  The Irminger Current, a warm subsurface current that constitutes part of the sub-polar gyre, plays an important role in "shaping" the spatial pattern of coastal melting; the amount of ice mass loss decreases along the pathway of the Irminger Current (from southeastern and southern and then southwestern Greenland).  The maximum northern extent of its influence in 2010 was about 69 \textordmasculine N.  

Chapter Four shows that accelerated melting of the Greenland ice sheet can in turn influence the regional ocean.  Recent freshwater flux from Greenland is estimated using GRACE gravity data.  Freshwater flux from Greenland, the Canadian Arctic Archipelago and the Arctic sea ice increase by 20 mSv from the mid-late 1990s to 2013.  I estimate that at least 70 percent of the increase winds up in the Labrador Sea due to the clockwise nature of ocean circulation around Greenland.  This study shows that a rapid decline in Labrador Sea Water thickness and density coincided with a rapid increase in freshwater flux into the Labrador Sea while the salt flux into the region remained high.  This suggests that recent accelerated melting of the Greenland ice sheet has started to reduce the formation of Labrador Sea Water and potentially weaken the Atlantic Meridional Overturning Circulation. 

Chapter Five shows that InSAR monitoring of surface deformation is a promising approach to estimate pressure changes in deep reservoirs subject to fluid injection.  Up to 10 cm surface uplift was observed between January 2007 and March 2011 at a CO$_{2}$-EOR field in Scurry County, West Texas. Monthly injection and production data and an analytical model are utilized to estimate the pressure change in the reservoir and to investigate causes of the observed uplift. Net CO$_{2}$ injection results in up to 12 MPa pressure build up in the reservoir, and was major contributor to the observed surface uplift.
 
\section{Future Work}
In Chapter Three, the GPS data are collected and processed up till 2011.  This study shows extreme ice mass loss in 2010, while later studies show that ice mass loss in 2012 and 2015 was also high \cite[e.g.,][]{hanna2014chpt6,tedesco2015chpt6}.  Continuous monitoring of annual variations in coastal uplift allows estimates of ice mass loss in coastal Greenland that are independent of GRACE.  Longer observations would allow for better correlation analysis between annual uplift and climatic factors.  Until now, only short-term elastic crustal response to current mass loss and long-term viscous crustal and upper mantle response to past ice mass loss have been considered as the controlling factor to vertical surface motion here.  Since recent accelerated mass loss of the Greenland ice sheet started three decades ago, the viscous response to this recent mass change should be considered \cite[e.g.,][]{nick2009chpt6,nield2014chpt6}.  KULU is a long-term GPS station located near Helheim Glacier in southeast Greenland.  The GPS time series here shows non-linear uplift since 2003.  The rapid speed up and a subsequent slowdown behavior are well known from various types of data for Helheim Glacier \cite[e.g.,][]{howat2005chpt6,nick2009chpt6}, which is probably a big source of the KULU uplift.  Thus, it is probably an over-simplification to fit a constant acceleration model to the KULU GPS time series.  A combination of various ice mass change record and a viscoelastic crustal/upper mantle response could better explain the non-linear uplift pattern at KULU.  

In Chapter Four, the freshwater flux estimate is a minimum estimate.  I  focused on three freshwater sources that likely influence Labrador Sea convection and can be estimated by remote techniques: the Greenland Ice Sheet, glaciers in the Canadian Arctic Archipelago, and changes in Arctic sea ice.  Other sources such as precipitation minus evaporation, oceanic transport and melt water from the annual freeze-thaw cycle of Arctic Sea ice also contribute to the Arctic freshwater budget.  Thus, besides remote techniques, in situ measurements are required to better measure the Arctic freshwater budget.  Chemical tracers can help to investigate the pathways of freshwater and distinguish freshwater of different origins \cite[]{haine2015chpt6}.  This study suggests that at least 70\% of Arctic freshwater flux ends up in the Labrador Sea, reducing the formation of Labrador Sea Water, hence weakening the AMOC.  This hypothesis can be validated by sophisticated numerical  experiments that allow freshwater flux to be focused into the Labrador Sea and determining its effect on the AMOC.  

In Chapter Five, I show that InSAR-monitored surface deformation can be an indicator of reservoir pressure change.  However, better knowledge of rheological parameters such as Young’s modulus is required to more quantitatively link surface deformation and reservoir pressure.  Thus, independent determination of Young’s modulus from down-hole measurements or other methods is suggested for future work.  

\section{References}
\bibliographystyle{apalike}  
\bibliography{chpt6_ref}