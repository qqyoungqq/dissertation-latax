\chapter{Introduction}
\section{Overview}
Satellite geodesy is the measurement of the size and shape of the earth as well as its gravity field by means of artificial satellites.  Satellite geodesy is a powerful tool to monitor time variations in the Earth related to plate tectonics, post-glacial rebounds, ocean circulation, ground water extraction, and a host of other natural and anthropogenic processes.  This dissertation focuses on the application of satellite geodesy to studies of environmental and global change.  Data from three techniques are used: high precision Global Positioning System (GPS), Interferometric Synthetic Aperture Radar (InSAR) and the Gravity Recovery and Climate Experiment (GRACE).  

The dissertation has six chapters. Chapter One (this chapter) is an introduction and summary of the work.  Chapter two describes the essentials of three satellite geodesy techniques.  The next three chapters (Chapter Three, Four and Five) are each based on a published, open literature paper.  

Chapter Three focuses on using coastal vertical displacement observed by high precision GPS to study recent mass loss of the Greenland ice sheet.  High precision GPS has been used to study a number of Earth processes, including plate motion, fault-related crustal deformation, and coastal subsidence.  Many of these applications involve looking at secular (long-term) rates of surface deformation, where the displacement rate can be assumed constant over the measurement period, typically several years or longer.  In a number of Earth processes, however, it is also useful to consider short-term fluctuations.  Many of these applications involve changes in Earth's fluid envelope, for example annual loading and unloading of the crust associated with the hydrologic cycle.  Accelerating uplift of the coastal regions of Greenland, where most of the current mass loss is concentrated \cite[e.g.,][]{zwally2005intro,thomas2006intro,luthcke2006intro,rignot2006intro,wouters2008intro}, is well recorded by a network of GPS stations emplaced on the rocky margins \cite[]{bevis2012intro}.  The previous study of \citet{jiang2010intro} focused on decadal scale trends and demonstrated that decadal time series of the vertical position component were well fit by a simple model of constant acceleration.  \citet{jiang2010intro} assumed constant amplitude of annual uplift each year, a common assumption in GPS time series analysis. However, the data show significant annual variation.  More recent measurements suggest that accelerating melting of Greenland ice sheet is continuing, with some melting seasons (for example 2010, 2012) experiencing significant ice mass loss \cite[]{bevis2012intro,nghiem2012intro}.  Thus, the short-term annual variation of coastal uplift measured by GPS can be useful in studying variable and accelerating ice mass loss.

One important aspect of the current retreat phase in Greenland is the role of climate forcing on melting coastal areas of Greenland.  In Greenland, ice mass change is regulated by two climate factors, atmospheric forcing \cite[]{zwally2002intro,hall2008intro} and oceanic forcing \cite[]{van2008intro,holland2008intro,hanna2009intro,straneo2010intro,straneo2012intro,straneo2013intro,seale2011intro}.  Atmospheric forcing can affect surface mass balance (SMB) by changing either or both the snow accumulation rate and the ablation rate. Also, melt water can influence the basal sliding rate. Oceanic forcing can increase submarine melting of marine-terminating outlet glaciers, resulting in rapid changes in calving rate, and inducing dynamic changes upstream, including glacier acceleration and thinning \cite[]{straneo2013intro}.  GRACE satellite data documents mass loss in Greenland over the last decade, and for West Greenland, clearly shows that loss is concentrated along the coast \cite[e.g.,][]{wouters2008intro}.  Unfortunately these data lack the spatial resolution to investigate melting at the scale of individual drainage basins.  However, coastal uplift as measured by GPS is sensitive to ice loss at this scale, which allows assessment of the influence of local climate conditions on melting.   In my dissertation, both short-term and long-term surface deformation processes measured by GPS is utilized to understand the climatic forcing on mass loss. What are the main driving forces for recent accelerated mass loss of Greenland Ice Sheet?  What is the relative contribution of oceanic versus atmospheric forcing on coastal melting?  Those questions will be discussed in my dissertation by using GPS data combined with other oceanographic and meteorological data.

Chapter Four uses GRACE data to estimate the recent freshwater flux from Greenland and investigates its impact on the Atlantic Meridional Overturning Circulation (AMOC).  The AMOC is a major mode of ocean thermo-haline circulation.  It is driven by density differences in the Atlantic Ocean, and is a key component of the global climate system.  Both theoretical and numerical studies show that the AMOC is sensitive to freshwater balance because of the strong influence on sea water density \cite[]{stommel1961intro,rooth1982intro,rahmstorf1995intro,stouffer2006intro}.  Past abrupt climate changes have been linked with changes in the AMOC in response to changes in the freshwater budget \cite[]{manabe1993intro,manabe1995intro,clark2002intro}.  Recent anthropogenic warming and accelerated melting of the Greenland ice sheet is leading to a general freshening of the North Atlantic, raising concerns that the AMOC may soon be disrupted.  

In this dissertation, I estimate Arctic freshwater flux from three sources that are undergoing rapid increases and can be estimated from remote observations: the Greenland ice sheet, CAA glaciers and Arctic sea ice.  Among these, freshwater flux from Greenland is the largest component, and estimated with GRACE data and RACMO2.3 model \cite[]{ettema2009intro,noel2015intro}.  The pattern of coastal currents around Greenland tends to focus coastal waters towards the Labrador Sea, an important ``incubator'' for dense, North Atlantic Deep Water (NADW).  Southward return flow of NADW is an important component of the AMOC, hence any disruption of density balance in the Labrador Sea may be a leading indicator of changes to the AMOC.  In this dissertation, I compare freshwater flux estimates to properties of Labrador Sea Water and suggest that increased freshwater flux here is starting to impact the AMOC.  

Chapter Five focuses on using surface deformation observed by InSAR to study reservoir pressure change caused by fluid injection and production at an enhanced oil recovery field.  Similar to GPS, InSAR has been used to study a number of Earth processes.  Particularly, it has been used to monitor ground subsidence associated with oil and gas extraction \cite[]{tomas2005intro}.  As oil reservoirs have been drawn down in the last few decades, producers have increasingly applied enhanced oil recovery (EOR) techniques to increase the amount of oil that can be extracted from a given oil field.  This usually involves pumping of CO$_{2}$ or saline water into the reservoir, and raising the reservoir pressure.  Similar techniques are used in ``fracking'' (hydraulic fracturing) to stimulate natural gas production, and later, to get rid of water.  In some regions, there is concern that rapid pumping of water fluids into deep reservoirs can stimulate induced seismicity \cite[e.g.,][]{keranen2013intro,mcgarr2015intro}.  Here, there is a need for research into the rock mechanical and fluid mechanical processes involved in such fluid pumping.  There is also interest in pumping CO$_{2}$ from industrial plants into deep geological formations for large-scale Carbon Capture and Storage (CSS), thereby reducing CO$_{2}$ emissions to the atmosphere.  Research is underway to study the geomechanical impact of CCS, including microseismicity, fault reactivation, fracturing and ground deformation \cite[e.g.,][]{streit2004intro,zhou2010intro,mazzoldi2012intro,rinaldi2013intro,vasco2010intro}.  Here, ground deformation associated with fluid injection and production is studied to better understand the links between surface deformation and pressure changes at depth.  A numerical model incorporating rock and fluid properties is constructed to relate surface deformation to pressure changes at depth.  My method offers an inexpensive way to monitor deep reservoir pressure change based on low cost commercial satellite imagery.  

Chapter Six summarizes the main conclusions from the previous chapters, and makes suggestions for future research.  

\section{References}
\bibliographystyle{apalike}  
\bibliography{intro_ref}